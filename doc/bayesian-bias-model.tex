\documentclass[12pt]{article}
\usepackage[utf8]{inputenc}
\usepackage[left=1in,right=1in,top=1in,bottom=1in]{geometry}
\usepackage{amsfonts,amssymb,amsmath}
\usepackage{graphicx}
\usepackage{float}
\usepackage{xcolor}
\usepackage{mathtools}
\usepackage{authblk}
% Caption package for formatting figure legends
\usepackage[font={footnotesize,sf}]{caption}
% Allow compiling of entire root document or individual sub-documents
\usepackage{subfiles}
% Enable double spacing commands
% \usepackage{setspace}
%\doublespacing
% \setstretch{1.4}
%% Citation and bibliography
\usepackage[backend=biber,citestyle=numeric-comp,bibstyle=authoryear,sorting=none,
    uniquename=false,uniquelist=false,date=year,
    %% Bibliography
    giveninits=true,url=false,doi=true,isbn=false,eprint=false,dashed=false,
    % Truncate papers with > maxbibnames authors to first minbibnames authors + et al.
    maxbibnames=10,minbibnames=5
]{biblatex}
\bibliography{references.bib}
% Add numbers to bibliography
% see https://tex.stackexchange.com/questions/95337/print-year-after-author-in-biblatex-numeric-style
\DeclareNameAlias{sortname}{first-last}
\DeclareFieldFormat{labelnumberwidth}{\mkbibbrackets{#1}}
\defbibenvironment{bibliography}
  {\list
     {\printtext[labelnumberwidth]{%
    \printfield{labelprefix}%    \printfield{prefixnumber}%
    \printfield{labelnumber}}}
     {\setlength{\labelwidth}{\labelnumberwidth}%
      \setlength{\leftmargin}{\labelwidth}%
      \setlength{\labelsep}{\biblabelsep}%
      \addtolength{\leftmargin}{\labelsep}%
      \setlength{\itemsep}{\bibitemsep}%
      \setlength{\parsep}{\bibparsep}}%
      \renewcommand*{\makelabel}[1]{\hss##1}}
  {\endlist}
  {\item}
% Get rid of quotes around title
\DeclareFieldFormat[article,inbook,incollection,inproceedings,patent,thesis,unpublished]{title}{#1}
% Suppress In: before journal name
\renewbibmacro{in:}{}
% Suppress pp. before page numbers
\DeclareFieldFormat[article]{pages}{#1}
% https://tex.stackexchange.com/questions/438396/regex-replacement-for-biblatex-biber-to-eliminate-the-warning-month-not-integer
\DeclareSourcemap{
  \maps[datatype=bibtex]{
    \map[overwrite]{
      \step[fieldsource=month, match=\regexp{\A(j|J)an\Z}, replace=1]
      \step[fieldsource=month, match=\regexp{\A(f|F)eb\Z}, replace=2]
      \step[fieldsource=month, match=\regexp{\A(m|M)ar\Z}, replace=3]
      \step[fieldsource=month, match=\regexp{\A(a|A)pr\Z}, replace=4]
      \step[fieldsource=month, match=\regexp{\A(m|M)ay\Z}, replace=5]
      \step[fieldsource=month, match=\regexp{\A(j|J)un\Z}, replace=6]
      \step[fieldsource=month, match=\regexp{\A(j|J)ul\Z}, replace=7]
      \step[fieldsource=month, match=\regexp{\A(a|A)ug\Z}, replace=8]
      \step[fieldsource=month, match=\regexp{\A(s|S)ep\Z}, replace=9]
      \step[fieldsource=month, match=\regexp{\A(o|O)ct\Z}, replace=10]
      \step[fieldsource=month, match=\regexp{\A(n|n)ov\Z}, replace=11]
      \step[fieldsource=month, match=\regexp{\A(d|D)ec\Z}, replace=12]
    }
  }
}

% For tables made with kableExtra
\usepackage{booktabs}
%% Clickable reference links and urls
\usepackage[hidelinks]{hyperref}
% \hypersetup{
%     colorlinks=true,
%     linkcolor=blue,
%     citecolor=.,
%     filecolor=magenta,      
%     urlcolor=blue,
% }
\urlstyle{same}

%% Remove section numbering
\makeatletter
\renewcommand{\@seccntformat}[1]{}
\makeatother
\renewcommand{\numberline}[1]{}

%% Custom macros
\DeclareMathOperator{\clo}{\mathcal{C}}
\DeclareMathOperator{\cen}{cen}
\DeclareMathOperator{\clr}{clr}
\DeclareMathOperator{\gm}{g}
\DeclareMathOperator{\E}{E}
\DeclareMathOperator{\var}{var}
\DeclareMathOperator{\mvar}{mvar}
\renewcommand*{\vec}[1]{\mathbf{#1}}
% multiply / divide
\DeclareMathOperator{\td}{\:{}^\times \! / \:}
% \renewcommand*{\vec}[1]{{#1}}
\renewcommand{\div}{\: / \:}
\newcommand{\sample}[1]{\text{S}_{#1}}
% Norm
\DeclarePairedDelimiter\norm{\lVert}{\rVert}
% Absolute value
\DeclarePairedDelimiter\abs{\lvert}{\rvert}


\begin{document}

\section{Probabilistic formulation of the bias model for Bayesian or
  maximum-likelihood inference}

\subsection{Controls samples with known true composition}

$K$ is the number of taxa.

For sample $s$, let $\vec A(s)$ be the vector of actual relative abundances.
Assume $\vec A(s)$ is known with absolute certainty. Let $\vec O(s)$ be the
vector of relative abundances we would observe if we sequenced with infinite
read depth.  Without loss of generality, take $\vec A(s)$ and $\vec O(s)$ to be
normalized to proportions, such that $\sum_i \vec A_i(s) = 1$ and $\sum_i \vec
O_i(s) = 1$.  Let $\vec C(s)$ be the vector of read counts that we actually
observe, and assume this is multinomial conditional on $\vec O(s)$ and the read
depth,
\begin{align}
  \vec C(s) \mid \sum_i C_i(s), \vec O(s) 
  \sim \mathrm{Multinomial}\left(\sum_i C_i(s), \vec O(s) \right),
\end{align}
where $\sum_i C_i(s)$ is total read count for all taxa in the sample. Note that
$\sim$ is here and throughout to mean ``is distributed as'' rather than to
denote compositional equivalence.

Take the bias $\vec B(s)$ in sample $s$ to be a random compositional vector
such that
\begin{align}
    \label{eq:model}
    \vec O(s) = \clo\left[ \vec A(s) \cdot \vec B(s) \right],
\end{align}
where $\clo(\vec X) = \vec X / \sum_i X_i$ denotes ``closure'' or normalization
to proportions.

The spirit of our model is that the compositional mean (or ``center'') of $\vec
B(s]$ or arithmetic mean of $\clr \vec B(s)$ is determined only by the protocol
$P$ and therefore constant across samples. Our primary goal is to estimate
either of these quantities, $\cen[\vec B(s)]$ or $E[\clr \vec B(s)]$, which are
connected via $\cen[\vec B(s)] = \exp\{ E[\clr \vec B(s)] \}$.

Let $\beta = E[\clr \vec B(s)]$ be the mean and $\vec \Sigma$ be the covariance
matrix of $\clr \vec B(s)$, which we assume to be independent of sample
composition, and take as multivariate normal (NVM),
\begin{align*}
  \clr \vec B(s) \sim N(\vec \beta, \vec \Sigma).
\end{align*}

\paragraph{Priors:}

I've been using the (hierarchical) priors
\begin{align}
  \label{eq:priors}
% O &\sim A \oplus LN(\vec \beta, \vec \Sigma)
% \\ C &\sim Multnomial\left(\sum C_k, O \right)
  \beta_i &\sim N(0, \sigma_\beta)
  \\ \sigma_\beta &\sim \mathrm{Exp}(3)
  \\ \vec \Sigma &= \mathrm{Diag}(\sigma_1, \dots, \sigma_K)
  \\ \sigma_i &\sim \mathrm{Exp}(1).
\end{align}

Specifying such a model in Stan is complicated by the fact that the $\beta_i$'s
should sum to $0$. I've been experimenting with the different strategies
described in Section 1.7 ``Parameterizing Centered Vectors'' of the Stan Users Guide
v2.19
\url{https://mc-stan.org/docs/2_19/stan-users-guide/parameterizing-centered-vectors.html}.

Based on biological considerations and the efficiencies we observe in the
Brooks and Costea datasets, a prior on $\beta_i$ with fatter tails is probably
more appropriate than the normal distribution.

\paragraph{Note about informatic bias following multinomial sampling:}

I'm supposing that the multinomial sampling correponds to the joint operation
of sequencing and read assignment, yielding a vector of counts $\vec C{(s)}$.
Many taxonomic profiling programs normalize $\vec C{(s)}$ by dividing by
taxon-specific factors (such as marker-gene copy number or length). This step
is essentially adding bias that wasn't included in our model above.  But since
this informatic normalization is a deterministic operation, we can simply add
it to our estimate of $\vec B$ (or $\vec \beta$), as long we have access to the
taxon-specific factors and have access to the counts $\vec C{(s)}$ (needed for
fitting the above model to begin with).

\subsection{Differential bias from paired samples with unknown composition}

\subsubsection{Model}

Strategy: Substitute $\vec O^{(P)}(s)$ for $\vec O(s)$ and $\vec O^{(R)}(s)$
for $\vec A(s)$; but since we observe $\vec C^{(R)}(s)$ rather than $\vec
O^{(R)}(s)$, we have to treat $\vec O^{(R)}(s)$ as a parameter as well.

As before, take the counts observed by protocol $P$ to be multinomial,
\begin{align}
  \label{eq:C-P}
  \vec C^{(P)}(s) \mid \sum_i C^{(P)}_i(s), \vec O^{(P)}(s) 
  \sim \mathrm{Multinomial}\left(\sum_i C^{(P)}_i(s), \vec O^{(P)}(s) \right),
\end{align}
We now put $\vec O^{(R)}$ in place of $\vec A(s)$,
\begin{align}
    \label{eq:O-P}
    \vec O^{(P)}(s) = \clo\left[ \vec O^{(R)}(s) \cdot \vec B^{(P/R)}(s) \right],
\end{align}
and model the differential bias as MVN,
\begin{align}
  \label{eq:distn-clrB-PR}
  \clr \vec B^{(P/R)}(s) \sim N(\vec \beta^{(P/R)}, \vec \Sigma^{(P/R)}),
\end{align}

The form is the same as before with $\vec O^{(R)}(s)$ substituted for $\vec
A(s)$. Since we don't directly observe $\vec O^{(R)}(s)$, we also assume the
counts seen by the reference protocol to be multinomial,
\begin{align}
  \label{eq:C-R}
  \vec C^{(R)}(s) \mid \sum_i C^{(R)}_i(s), \vec O^{(R)}(s) 
  \sim \mathrm{Multinomial}\left(\sum_i C^{(R)}_i(s), \vec O^{(R)}(s) \right).
\end{align}
We now also need a prior on $\vec O^{(R)}(s)$, which I've been taking to be
$\mathrm{Dirichlet}(0.5)$.

\subsubsection{Details}

The likelihood model can be derived by supposing that, conditional on $\vec
A(s)$, the observations from each protocol individually follow the model of the
previous section. Starting with a model in which the observed counts of each
protocol depend on $\vec A(s)$, we are able to obtain a model without reference
to $\vec A(s)$, thanks to 1) differential bias equalling the difference in the
bias of the individual protocols and 2) the simple algebra of differences of
independent MVNs.

Suppose for each sample $s$ we have data $\vec C^{(P)}(s)$ and 
$\vec C^{(R)}(s)$ for protocols $P$ and $R$. Again take the counts to be
multinomial given the ``observed'' proportions $\vec O^{(P)}(s)$ and $\vec
O^{(R)}(s)$, which equal
\begin{align}
    \label{eq:O-P}
    \vec O^{(P)}(s) = \clo\left[ \vec A(s) \cdot \vec B^{(P)}(s) \right]
\end{align}
and 
\begin{align}
    \label{eq:O-R}
    \vec O^{(R)}(s) = \clo\left[ \vec A(s) \cdot \vec B^{(R)}(s) \right].
\end{align}

As before, we assume that the CLR bias of each sample is independent
MVN, though now with protocol specific mean and variances,
\begin{align}
  \label{eq:distn-clrB-P}
  \clr \vec B^{(P)}(s) \sim N(\vec \beta^{(P)}, \vec \Sigma^{(P)})
\end{align}
and
\begin{align}
  \label{eq:distn-clrB-R}
  \clr \vec B^{(R)}(s) \sim N(\vec \beta^{(R)}, \vec \Sigma^{(R)}).
\end{align}
It follows that the CLR differential bias, 
\begin{align} 
  \label{}
  \clr \vec B^{(P/R)}(s) &= \clr \left[\vec B^{(P)}(s) / \vec B^{(R)}(s)]
  \\&= \clr \vec B^{(P)}(s) - \clr \vec B^{(R)}(s),
\end{align}
is also MVN,
\begin{align}
  \label{eq:distn-clrB-PR}
  \clr \vec B^{(P/R)}(s) \sim N(\vec \beta^{(P/R)}, \vec \Sigma^{(P/R)}),
\end{align}
with mean
\begin{align}
  \label{eq:beta-PR}
  \vec \beta^{(P/R)} = \vec \beta^{(P)} - \vec \beta^{(P)}
\end{align}
and covariance
\begin{align}
  \label{eq:Sigma-PR}
  \vec \Sigma^{(P/R)} = \vec \Sigma^{(P)} + \vec \Sigma^{(R)}.
\end{align}
So we can simply estimate $\vec \beta^{(P/R)}$ and $\vec \Sigma^{(P/R)}$
directly using the likelihood from \eqref{eq:distn-clrB-PR}.

\section{Questions and issues}

Question: how to pin down $\vec B$ and its mean and (co)variance?

Question: how does the choice of log-ratio transform change the interpretation
of the parameters to the normal distribution (in particular, of how we should
set priors).

In reality, we don't know $\vec A$ with certainty, and I don't think the
measurement error in $\vec A$ is going to be well captured by the log-normal
error model.

In many cases, the counts are normalized (e.g., by predicted marker gene length or copy
number); this bias is deterministic though, and so can be easily incorporated
afterwards if we know the counts vector and the normalization vector. Users
often will not though, since many pipelines do this automatically and don't
give the users easy acccess to these important quantities

Implicit or explicit filtering of taxa with low observed abundance typically
occurs in advance of the user obtaining $\vec C$ means that it will not be
truly multinomial.


todo
- add (P) decorations to the non-differential bias est section
- replace (P) equations in the diff bias section w/ eq refs to above
- replace duplicated eq'ns in the diff bias details section w/ refs to eqs
above
- revise test, adding missing info along the way.

\end{document}
