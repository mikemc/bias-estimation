\documentclass[12pt]{article}
\usepackage[utf8]{inputenc}
\usepackage[left=1in,right=1in,top=1in,bottom=1in]{geometry}
\usepackage{amsfonts,amssymb,amsmath}
\usepackage{graphicx}
\usepackage{float}
\usepackage{xcolor}
\usepackage{mathtools}
\usepackage{authblk}
% Caption package for formatting figure legends
\usepackage[font={footnotesize,sf}]{caption}
% Allow compiling of entire root document or individual sub-documents
\usepackage{subfiles}
% Enable double spacing commands
% \usepackage{setspace}
%\doublespacing
% \setstretch{1.4}
%% Citation and bibliography
\usepackage[backend=biber,citestyle=numeric-comp,bibstyle=authoryear,sorting=none,
    uniquename=false,uniquelist=false,date=year,
    %% Bibliography
    giveninits=true,url=false,doi=true,isbn=false,eprint=false,dashed=false,
    % Truncate papers with > maxbibnames authors to first minbibnames authors + et al.
    maxbibnames=10,minbibnames=5
]{biblatex}
\bibliography{references.bib}
% Add numbers to bibliography
% see https://tex.stackexchange.com/questions/95337/print-year-after-author-in-biblatex-numeric-style
\DeclareNameAlias{sortname}{first-last}
\DeclareFieldFormat{labelnumberwidth}{\mkbibbrackets{#1}}
\defbibenvironment{bibliography}
  {\list
     {\printtext[labelnumberwidth]{%
    \printfield{labelprefix}%    \printfield{prefixnumber}%
    \printfield{labelnumber}}}
     {\setlength{\labelwidth}{\labelnumberwidth}%
      \setlength{\leftmargin}{\labelwidth}%
      \setlength{\labelsep}{\biblabelsep}%
      \addtolength{\leftmargin}{\labelsep}%
      \setlength{\itemsep}{\bibitemsep}%
      \setlength{\parsep}{\bibparsep}}%
      \renewcommand*{\makelabel}[1]{\hss##1}}
  {\endlist}
  {\item}
% Get rid of quotes around title
\DeclareFieldFormat[article,inbook,incollection,inproceedings,patent,thesis,unpublished]{title}{#1}
% Suppress In: before journal name
\renewbibmacro{in:}{}
% Suppress pp. before page numbers
\DeclareFieldFormat[article]{pages}{#1}
% https://tex.stackexchange.com/questions/438396/regex-replacement-for-biblatex-biber-to-eliminate-the-warning-month-not-integer
\DeclareSourcemap{
  \maps[datatype=bibtex]{
    \map[overwrite]{
      \step[fieldsource=month, match=\regexp{\A(j|J)an\Z}, replace=1]
      \step[fieldsource=month, match=\regexp{\A(f|F)eb\Z}, replace=2]
      \step[fieldsource=month, match=\regexp{\A(m|M)ar\Z}, replace=3]
      \step[fieldsource=month, match=\regexp{\A(a|A)pr\Z}, replace=4]
      \step[fieldsource=month, match=\regexp{\A(m|M)ay\Z}, replace=5]
      \step[fieldsource=month, match=\regexp{\A(j|J)un\Z}, replace=6]
      \step[fieldsource=month, match=\regexp{\A(j|J)ul\Z}, replace=7]
      \step[fieldsource=month, match=\regexp{\A(a|A)ug\Z}, replace=8]
      \step[fieldsource=month, match=\regexp{\A(s|S)ep\Z}, replace=9]
      \step[fieldsource=month, match=\regexp{\A(o|O)ct\Z}, replace=10]
      \step[fieldsource=month, match=\regexp{\A(n|n)ov\Z}, replace=11]
      \step[fieldsource=month, match=\regexp{\A(d|D)ec\Z}, replace=12]
    }
  }
}

% For tables made with kableExtra
\usepackage{booktabs}
%% Clickable reference links and urls
\usepackage[hidelinks]{hyperref}
% \hypersetup{
%     colorlinks=true,
%     linkcolor=blue,
%     citecolor=.,
%     filecolor=magenta,      
%     urlcolor=blue,
% }
\urlstyle{same}

%% Remove section numbering
\makeatletter
\renewcommand{\@seccntformat}[1]{}
\makeatother
\renewcommand{\numberline}[1]{}

%% Custom macros
\DeclareMathOperator{\clo}{\mathcal{C}}
\DeclareMathOperator{\cen}{cen}
\DeclareMathOperator{\clr}{clr}
\DeclareMathOperator{\gm}{g}
\DeclareMathOperator{\E}{E}
\DeclareMathOperator{\var}{var}
\DeclareMathOperator{\mvar}{mvar}
\renewcommand*{\vec}[1]{\mathbf{#1}}
% multiply / divide
\DeclareMathOperator{\td}{\:{}^\times \! / \:}
% \renewcommand*{\vec}[1]{{#1}}
\renewcommand{\div}{\: / \:}
\newcommand{\sample}[1]{\text{S}_{#1}}
% Norm
\DeclarePairedDelimiter\norm{\lVert}{\rVert}
% Absolute value
\DeclarePairedDelimiter\abs{\lvert}{\rvert}


\author{Michael McLaren}
\title{Probabilistic formulation of the bias\\
  and differential-bias model for Bayesian inference}

\begin{document}

\maketitle

Notation generally follows that of the manuscript. Let $K$ be the number of
taxa of interest and $S$ be the set of samples. For sample $s$, let $\vec A(s)$
be the vector of actual relative abundances in the sample, let $\vec O(s)$ be
the vector of relative abundances in the sequencing library, and let $\vec
C(s)$ be the vector of read counts that we observe after sequencing and
bioinformatics. The sum $\sum_i C_i(s)$ denotes the read depth of sample $s$
for the $K$ taxa. (All sums over taxa indexed by $i$ are from $i=1$ to $i=K$ if
not explicitly noted.) Without loss of generality, take $\vec A(s)$ and $\vec
O(s)$ to be normalized to proportions, such that $\sum \vec A_i(s) = 1$
and $\sum_i \vec O_i(s) = 1$.  When we are considering more than one protocol,
we use a superscript $(P)$ to denote the protocol $P$, as in $\vec O^{(P)}(s)$
and $\vec C^{(R)}(s)$.

\subsection{Controls samples with known true composition}

\paragraph{Model:}

Assume we know the actual compositions $\vec A(s)$ is known with certainty
(i.e., ignoring any measurement error or sample-construction error).  Assume
that the observed read counts are multinomial conditional on $\vec O(s)$ and
the read depth; that is,
\begin{align}
  \vec C(s) \mid \sum_i C_i(s), \vec O(s) 
  \sim \mathrm{Multinomial}\left(\sum_i C_i(s), \vec O(s) \right),
\end{align}
Note that I used $\sim$ here and throughout to mean ``is distributed as''
rather than to denote compositional equivalence.
Let $\vec B(s)$ be the (compositional) vector of $K$ relative efficiencies in
sample $s$, such that
\begin{align} 
  \label{}
  \vec O(s) = \clo\left[\vec A(s) \cdot \vec B(s)\right],
\end{align}
where $\clo(\cdot)$ denotes the ``closure'' operation that converts a
compositional vector to a vector of proportions.
% In the model of Brian Williamson et al (2019), the vector $\vec B$
% is constant and given by the vector $\vec e$. A generalization is to take $\vec
% B(s)$ to be a random compositional vector following a given distribution. 
I take $\clr \vec B(s)$ to have a multivariate normal distribution (MVN;
assumed iid across samples). Let $\beta = E[\clr \vec B(s)]$ be the mean and
$\vec \Sigma$ be the covariance matrix of $\clr \vec B(s)$, so that
\begin{align*}
  \clr \vec B(s) \sim N(\vec \beta, \vec \Sigma).
\end{align*}
% \begin{align}
%     \label{eq:model}
%     \vec O(s) = \clo\left[ \vec A(s) \cdot \vec B(s) \right],
% \end{align}
% where $\clo(\vec X) = \vec X / \sum_i X_i$ denotes ``closure'' or normalization
% to proportions.
In my testing so far I have taken $\Sigma$ to be diagonal, due to inexperience
with covariance models (in particular, choosing sensible priors and coding the
model in Stan).

% TODO : explain issues regarding missing taxa

% Biologically, a diagonal matrix model reflects a scenario where there are many
% steps, and each step in the protocol introduces uncorrelated random noise. The
% products of uncorrelated noise across steps leads to the given log-normal
% distribution on the efficiencies across samples.

% The spirit of our model is that the compositional mean (or ``center'') of $\vec
% B(s]$ or arithmetic mean of $\clr \vec B(s)$ is determined only by the protocol
% $P$ and therefore constant across samples. Our primary goal is to estimate
% either of these quantities, $\cen[\vec B(s)]$ or $E[\clr \vec B(s)]$, which are
% connected via $\cen[\vec B(s)] = \exp\{ E[\clr \vec B(s)] \}$.

\paragraph{Priors:}

I've been using the multilevel priors for the Brooks dataset,
\begin{align}
  \label{eq:priors}
% O &\sim A \oplus LN(\vec \beta, \vec \Sigma)
% \\ C &\sim Multnomial\left(\sum C_k, O \right)
  \beta_i &\sim N(0, \sigma_\beta)
  \\ \sigma_\beta &\sim \mathrm{Exp}(3)
  \\ \vec \Sigma &= \mathrm{Diag}(\sigma_1, \dots, \sigma_K)
  \\ \sigma_i &\sim \mathrm{Exp}(1).
\end{align}

Specifying such a model in Stan is complicated by the fact that the $\beta_i$'s
must sum to $0$ as I've defined $\vec \beta$. I've been experimenting with the
different strategies described in Section 1.7 ``Parameterizing Centered
Vectors'' of the Stan Users Guide v2.19
\url{https://mc-stan.org/docs/2_19/stan-users-guide/parameterizing-centered-vectors.html}.
% I've also used an approach taken in Williamson, Hughes, and Willis (2019) would
% give $\vec \beta$ an different interpretation but mak

Based on biological considerations and the efficiencies we observe in the
Brooks and Costea datasets, a prior on $\beta_i$ with fatter tails is probably
more appropriate than the normal distribution.

\subsection{Differential bias from paired samples with unknown composition}

The starting point is to suppose that conditional on $\vec A(s)$, the
observations by two protocols $P$ and $R$ of sample $s$ each follow the model
of the previous section, with protocol-specific MVN distributions for $\clr
\vec B^{(P)}(s)$ and $\clr \vec B^{(R)}(s)$. 
Let $\vec B^{(P/R)}(s)$ be the (compositional) vector of $K$ differential
relative efficiencies in sample $s$, such that
\begin{align} 
  \label{}
  % \vec O(s) = \clo\left[\vec A(s) \cdot \vec B^{(P/R)}(s)],
  \vec O^{(P)}(s) = \clo\left[\vec O^{(R)}(s) \cdot \vec B^{(P/R)}(s)].
\end{align}
Since
\begin{align} 
  \label{eq:clr-BPR}
  \clr \vec B^{(P/R)}(s) = \clr \vec B^{(P)}(s) - \clr \vec B^{(R)}(s),
\end{align}
the distribution of the $\clr$ of the differential-bias vector $\vec
B^{(P/R)}(s)$ will also be MVN,
\begin{align}
  \label{eq:distn-clrB-PR}
  \clr \vec B^{(P/R)}(s) \sim N(\vec \beta^{(P/R)}, \vec \Sigma^{(P/R)}),
\end{align}
with $\vec \beta^{(P/R)} = \vec \beta^{(P)} - \vec \beta^{(R)}$. If the
measurement by each protocol is independent then we also have $\vec
\Sigma^{(P/R)} = \vec \Sigma^{(P)} + \vec \Sigma^{(R)}$. (It might not be
if the efficiencies depend on unmeasured sample-specific properties such as
pH.)

\paragraph{Model 1:}

Model 1 takes advantage of the relationship \eqref{eq:clr-BPR} to only model
$\clr \vec B^{(P/R)}$ and not the efficiencies of each protocol.  The counts
observed by each protocol $P$ are multinomial given the library proportions and
total read depths,
\begin{align}
  \label{eq:C-P}
  \vec C^{(P)}(s) \mid \sum_i C^{(P)}_i(s), \vec O^{(P)}(s) 
  \sim \mathrm{Multinomial}\left(\sum_i C^{(P)}_i(s), \vec O^{(P)}(s) \right),
\end{align}
\begin{align}
  \label{eq:C-R}
  \vec C^{(R)}(s) \mid \sum_i C^{(R)}_i(s), \vec O^{(R)}(s) 
  \sim \mathrm{Multinomial}\left(\sum_i C^{(R)}_i(s), \vec O^{(R)}(s) \right).
\end{align}
Model the clr of the differential relative efficiencies as MVN,
\begin{align}
  \label{eq:distn-clrB-PR}
  \clr \vec B^{(P/R)}(s) \sim N(\vec \beta^{(P/R)}, \vec \Sigma^{(P/R)}).
\end{align}
We also need a prior on $\vec O^{(R)}(s)$ (or $\vec O^{(R)}(s)$), which I've
been taking to be $\mathrm{Dirichlet}(0.5)$ during initial testing, but perhaps
an abundance-distribution model such as that used in with of Williamson,
Hughes, and Willis (2019) is more appropriate for complex natural communities.

\paragraph{Model 2:}

In Model 1, there is an asymmetry in having to choose one protocol to place a
prior on the proportions in its sequencing library (the $\vec O^{(R)}(s)$). We
might instead include $\vec A(s)$, $\vec B^{(P)}(s)$, and $\vec B^{(P)}(s)$ as
parameters. Then we could specify the library proportions as 
\begin{align} 
  \label{}
  \vec O^{(P)}(s) = \clo\left[\vec A(s) \cdot \vec B^{(P)}(s)]
\end{align}
and
\begin{align} 
  \label{}
  \vec O^{(R)}(s) = \clo\left[\vec A(s) \cdot \vec B^{(R)}(s)],
\end{align}
and place a prior on $\vec A(s)$. We would still only be interested in $\vec
B^{(P/R)}(s)$ (which we can compute from \eqref{eq:clr-BPR}), but specifying
the model in this way avoids the asymmetry. However, my intuition is that it
would be much slower to fit such a model due to the data providing being so
little information about $\vec A(s)$ (and hence $\vec B^{(P)}(s)$ and $\vec
B^{(R)}(s)$).

% which should have much less posterior uncertainty than $\vec B^{(P)}(s)$, and
% $\vec B^{(P)}(s)$

% \section{Questions / issues / notes}
%
% Question: how to pin down $\vec B$ and its mean and (co)variance?
%
% Question: how does the choice of log-ratio transform change the interpretation
% of the parameters to the normal distribution (in particular, of how we should
% set priors).
%
% In reality, we don't know $\vec A$ with certainty. (E.g., measurement error in
% $\vec A$ is plausibly the dominant source of noise in the Brooks PCR-mixture
% experiment.) I don't think the measurement error in $\vec A$ is going to be well
% captured by the log-normal error model.
%
% In many cases, the counts are normalized (e.g., by predicted marker gene length or copy
% number); this bias is deterministic though, and so can be easily incorporated
% afterwards if we know the counts vector and the normalization vector. Users
% often will not though, since many pipelines do this automatically and don't
% give the users easy acccess to these important quantities
%
% Implicit or explicit filtering of taxa with low observed abundance typically
% occurs in advance of the user obtaining $\vec C$ means that it will not be
% truly multinomial.
%
%
% todo
% - add (P) decorations to the non-differential bias est section
% - replace (P) equations in the diff bias section w/ eq refs to above
% - replace duplicated eq'ns in the diff bias details section w/ refs to eqs
% above
% - revise test, adding missing info along the way.
%
% \paragraph{Note about informatic bias following multinomial sampling:}
%
% I'm supposing that the multinomial sampling correponds to the joint operation
% of sequencing and read assignment, yielding a vector of counts $\vec C{(s)}$.
% Many taxonomic profiling programs normalize $\vec C{(s)}$ by dividing by
% taxon-specific factors (such as marker-gene copy number or length). This step
% is essentially adding bias that wasn't included in our model above.  But since
% this informatic normalization is a deterministic operation, we can simply add
% it to our estimate of $\vec B$ (or $\vec \beta$), as long we have access to the
% taxon-specific factors and have access to the counts $\vec C{(s)}$ (needed for
% fitting the above model to begin with).
%
%
\end{document}
