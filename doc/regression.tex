\documentclass[12pt]{article}
\usepackage[utf8]{inputenc}
\usepackage[left=1in,right=1in,top=1in,bottom=1in]{geometry}
\usepackage{amsfonts,amssymb,amsmath}
\usepackage{graphicx}
\usepackage{float}
\usepackage{xcolor}
\usepackage{mathtools}
\usepackage{authblk}
% Caption package for formatting figure legends
\usepackage[font={footnotesize,sf}]{caption}
% Allow compiling of entire root document or individual sub-documents
\usepackage{subfiles}
% Enable double spacing commands
% \usepackage{setspace}
%\doublespacing
% \setstretch{1.4}
%% Citation and bibliography
\usepackage[backend=biber,citestyle=numeric-comp,bibstyle=authoryear,sorting=none,
    uniquename=false,uniquelist=false,date=year,
    %% Bibliography
    giveninits=true,url=false,doi=true,isbn=false,eprint=false,dashed=false,
    % Truncate papers with > maxbibnames authors to first minbibnames authors + et al.
    maxbibnames=10,minbibnames=5
]{biblatex}
\bibliography{references.bib}
% Add numbers to bibliography
% see https://tex.stackexchange.com/questions/95337/print-year-after-author-in-biblatex-numeric-style
\DeclareNameAlias{sortname}{first-last}
\DeclareFieldFormat{labelnumberwidth}{\mkbibbrackets{#1}}
\defbibenvironment{bibliography}
  {\list
     {\printtext[labelnumberwidth]{%
    \printfield{labelprefix}%    \printfield{prefixnumber}%
    \printfield{labelnumber}}}
     {\setlength{\labelwidth}{\labelnumberwidth}%
      \setlength{\leftmargin}{\labelwidth}%
      \setlength{\labelsep}{\biblabelsep}%
      \addtolength{\leftmargin}{\labelsep}%
      \setlength{\itemsep}{\bibitemsep}%
      \setlength{\parsep}{\bibparsep}}%
      \renewcommand*{\makelabel}[1]{\hss##1}}
  {\endlist}
  {\item}
% Get rid of quotes around title
\DeclareFieldFormat[article,inbook,incollection,inproceedings,patent,thesis,unpublished]{title}{#1}
% Suppress In: before journal name
\renewbibmacro{in:}{}
% Suppress pp. before page numbers
\DeclareFieldFormat[article]{pages}{#1}
% https://tex.stackexchange.com/questions/438396/regex-replacement-for-biblatex-biber-to-eliminate-the-warning-month-not-integer
\DeclareSourcemap{
  \maps[datatype=bibtex]{
    \map[overwrite]{
      \step[fieldsource=month, match=\regexp{\A(j|J)an\Z}, replace=1]
      \step[fieldsource=month, match=\regexp{\A(f|F)eb\Z}, replace=2]
      \step[fieldsource=month, match=\regexp{\A(m|M)ar\Z}, replace=3]
      \step[fieldsource=month, match=\regexp{\A(a|A)pr\Z}, replace=4]
      \step[fieldsource=month, match=\regexp{\A(m|M)ay\Z}, replace=5]
      \step[fieldsource=month, match=\regexp{\A(j|J)un\Z}, replace=6]
      \step[fieldsource=month, match=\regexp{\A(j|J)ul\Z}, replace=7]
      \step[fieldsource=month, match=\regexp{\A(a|A)ug\Z}, replace=8]
      \step[fieldsource=month, match=\regexp{\A(s|S)ep\Z}, replace=9]
      \step[fieldsource=month, match=\regexp{\A(o|O)ct\Z}, replace=10]
      \step[fieldsource=month, match=\regexp{\A(n|n)ov\Z}, replace=11]
      \step[fieldsource=month, match=\regexp{\A(d|D)ec\Z}, replace=12]
    }
  }
}

% For tables made with kableExtra
\usepackage{booktabs}
%% Clickable reference links and urls
\usepackage[hidelinks]{hyperref}
% \hypersetup{
%     colorlinks=true,
%     linkcolor=blue,
%     citecolor=.,
%     filecolor=magenta,      
%     urlcolor=blue,
% }
\urlstyle{same}

%% Remove section numbering
\makeatletter
\renewcommand{\@seccntformat}[1]{}
\makeatother
\renewcommand{\numberline}[1]{}

%% Custom macros
\DeclareMathOperator{\clo}{\mathcal{C}}
\DeclareMathOperator{\cen}{cen}
\DeclareMathOperator{\clr}{clr}
\DeclareMathOperator{\gm}{g}
\DeclareMathOperator{\E}{E}
\DeclareMathOperator{\var}{var}
\DeclareMathOperator{\mvar}{mvar}
\renewcommand*{\vec}[1]{\mathbf{#1}}
% multiply / divide
\DeclareMathOperator{\td}{\:{}^\times \! / \:}
% \renewcommand*{\vec}[1]{{#1}}
\renewcommand{\div}{\: / \:}
\newcommand{\sample}[1]{\text{S}_{#1}}
% Norm
\DeclarePairedDelimiter\norm{\lVert}{\rVert}
% Absolute value
\DeclarePairedDelimiter\abs{\lvert}{\rvert}


\DeclareMathOperator{\vecop}{vec}

\author{Michael McLaren, m.mclaren42@gmail.com}

\begin{document}

\maketitle

% todo
% - what I've done so far
% - version with transposed mat's
% - version with taxa covariates

\section{Bias and linear regression: Least-squares solution}

For these notes, I'm going to be sticking to the log scale and using
traditional multivariate regression matrix notation.  Following the notation of
``Matrix Algebra'' (Abadir and Magnus), I use $\vecop(\cdot)$ for the matrix
vectorization operator, $\vec X'$ for the transpose of $\vec X$, $\vec X^+$ for
the Moore-Penrose inverse of $\vec X$, and $\vec X_{i} = \vec X_{i\cdot}$ to be
a column vector whose elements are the $i$-th row of $\vec X$.

Let $n$ be the number of samples, $p$ be the number of sample covariates, and
$q$ be the number of taxa. 
Let $\vec Y$ be the $n \times q$ matrix of log compositional errors that we
observed for the $n$ samples, with rows corresponding to samples and columns
corresponding to taxa. In terms of the manuscript's notation, $\vec Y_{i} =
\vec Y_{i\cdot} = \ln [\vec O(i) / \vec A(i)]$.  Notably,
\begin{enumerate}
  \item The meaning of $\vec Y_{i}$ is only contained in the differences
    between elements (taxa) and would be unchanged if each $\vec Y_i$ were
    multiplied by a positive constant. 
  \item The elements of $\vec Y_i$ corresponding to missing taxa ($0/0$) are
    interpreted as providing no information about the difference between that
    element and the other elements of $\vec Y_i$.
\end{enumerate}
Let $\vec P^{(i)}$ be the $q\times q$ clr-projection matrix, which projects
$\vec Y_{i}$ onto the subspace of the clr vector space corresponding to the
taxa that are present in sample $i$. To project $\vec Y_{i}$ by
left-multiplication by $\vec P^{(i)}$, we first must replace the $0/0$s with
(arbitrary) finite values. Then, the transformation $\vec Y_{i}\mapsto \vec
P^{(i)}\vec Y_{i}$ centers the non-missing elements to sum to 0 by subtracting
their mean, and sets the missing elements to 0.

Let $\vec X$ be the $n\times(p+1)$ matrix of sample covariates and $\vec B$ be
the $(p+1)\times q$ matrix of log-efficiency parameters which it is our goal to
learn. To pin $\vec B$ down to a well-defined value, assume that the rows of
$\vec B$ are each centered to sum to 0. Also, let $\vec E$ be the $n \times q$
matrix of residual (random) errors. The expected log efficiencies for sample
$i$ are $\vec X_i' \vec B$, and the realized log efficiencies are
$\vec X_i' \vec B + \vec E_i'$.

If all taxa were present in all samples, we could simply take all the
log-compositions to be centered and write the model for sample $i$ as $\vec
Y_i' = \vec X_i' \vec B + \vec E_i'$ and for all samples as 
\begin{align} 
  \label{eq:model-simple}
  \vec Y = \vec X \vec B + \vec E.
\end{align}
The least-squares estimate for $\vec B$ would then simply be
\begin{align}
  \label{eq:lss-simple}
  \hat{\vec B} = (\vec X' \vec X)^{+} \vec X' \vec Y.
\end{align}

With missing taxa, the model for sample $i$ is instead
\begin{align} 
  \label{}
  \vec P^{(i)} \vec Y_i 
  % &= \vec P^{(i)} [ (\vec X_i' \vec B)' + \vec E_i ]
  &= \vec P^{(i)} [ \vec B' \vec X_i + \vec E_i ]
  \\&= \vec P^{(i)} \vec B' \vec X_i + \vec P^{(i)}\vec E_i.
\end{align}
Since the projection matrices differ across samples, we can no longer simply
center all the samples identically to satisfy \eqref{eq:model-simple}.  Our
strategy is instead to use vectorization and construction of a block diagonal
matrix from the projection matrices $\vec P^{(i)}$ to obtain an equation in the
form \eqref{eq:model-simple}, so that we can employ \eqref{eq:lss-simple}.

Let $\tilde{\vec P} = \text{diag}\{\vec P^{(i)}\}$ be the $nq \times nq$
block-diagonal matrix created from the list of matrices $\{\vec P^{(i)}\}$.  To
project all samples onto their relevant subspaces, we multiply $\tilde{\vec
  P}$ by $\vecop(\vec Y')$, where the log-compositional error of sample $i$ is
contained in rows $(i-1)q + 1$ to $iq$ of $\vecop(\vec Y')$.
% That is,
% \begin{align*} 
%   \tilde{\vec P} \vecop(\vec Y') = 
%     \begin{bmatrix}
%       \vec P^{(1)} \vec Y_1\\
%       \vdots \\
%       \vec P^{(n)} \vec Y_n
%     \end{bmatrix}.
% \end{align*}
Now we can write the linear model for all samples as
\begin{align} 
  \tilde{\vec P} \vecop(\vec Y') 
    &= \tilde{\vec P} \vecop(\vec B' \vec X') + \tilde{\vec P} \vecop(\vec E'),
\end{align}
which has the expanded form
\begin{align*} 
  \begin{bmatrix}
    \vec P^{(1)} \vec Y_1\\
    \vdots \\
    \vec P^{(n)} \vec Y_n
  \end{bmatrix}
  = 
  \begin{bmatrix}
    \vec P^{(1)} \vec B' \vec X_1\\
    \vdots \\
    \vec P^{(n)} \vec B' \vec X_n
  \end{bmatrix}
  +
  \begin{bmatrix}
    \vec P^{(1)} \vec E_1\\
    \vdots \\
    \vec P^{(n)} \vec E_n
  \end{bmatrix}.
\end{align*}
Next, we use an identity involving the Kronecker product and the vec operator
(Abadir and Magnus, \emph{Matrix Algebra}, p. 282) to obtain $\vecop(\vec B'
\vec X') = (\vec X \otimes \vec I_q) \vecop(\vec B')$, and hence
\begin{align} 
  \label{}
  \tilde{\vec P} \vecop(\vec Y') 
    % &= \tilde{\vec P} \vecop(\vec (XB)') + \tilde{\vec P} \vecop(\vec E')
  &= \tilde{\vec P} (\vec X \otimes \vec I_q) \vecop(\vec B')
    + \tilde{\vec P} \vecop(\vec E').
\end{align}
We now have the desired form, which can be solved by the standard least-squares
solution.
Let
\begin{itemize}
  \item $\tilde{\vec y} = \tilde{\vec P} \vecop(\vec Y')$
  \item $\tilde{\vec X} = \tilde{\vec P} (\vec X \otimes \vec I_q)$
  \item $\tilde{\vec b} = \vecop(\vec B')$
  \item $\tilde{\vec e} = \tilde{\vec P} \vecop(\vec E')$
\end{itemize}
so that
\begin{align} 
  \label{}
  \tilde{\vec y} &= \tilde{\vec X} \tilde{\vec b} + \tilde{\vec e}.
\end{align}
The least-squares solution for $\tilde{\vec b} = \vecop (\vec B')$ is
\begin{align}
  \label{eq:b-tilde-hat}
  % \vecop(\hat{\vec B}')
  \hat{\tilde{\vec b}}
  = (\tilde{\vec X}' \tilde{\vec X})^{+} \tilde{\vec X}' \tilde{\vec y}.
\end{align}

\subsubsection{Special case of no covariates}

If there are no sample covariates ($p=0$), then we are in the situation
considered in Mclaren2019. In this case, we expect that the solution
\eqref{eq:b-tilde-hat} should equate to the estimator described by
vandenBoogaart2006, which in the present notation we can write as
\begin{align} 
  \label{}
  \hat {\vec B}_1 = \left(\sum_i \vec P^{(i)} \right)^+ 
    \sum_i \vec P^{(i)} \vec Y_i.
\end{align}
(Here, $\vec B$ is a $1\times q$ matrix and so $\vec B}_1$ is just a way to
write the column vector of log efficiencies.)
We can confirm this equivalence by noting that 
\begin{align*} 
  \tilde {\vec y} = 
    \begin{bmatrix}
      \vec P^{(1)} \vec Y_1\\
      \vdots \\
      \vec P^{(n)} \vec Y_n
    \end{bmatrix},
\end{align*}
and that for this special case we have that
\begin{align*} 
  \tilde {\vec X} = 
    \begin{bmatrix}
      \vec P^{(1)}\\
      \vdots \\
      \vec P^{(n)}
    \end{bmatrix}
\end{align*}
and
\begin{align*} 
  \tilde {\vec b} = \vec B_1.
\end{align*}
It follows that
\begin{align*} 
  \tilde {\vec X}' \tilde {\vec X}
  = \sum_i \vec P^{(i)} \vec P^{(i)}
  = \sum_i \vec P^{(i)}
\end{align*}
and
\begin{align*} 
  \tilde {\vec X}' \tilde {\vec y}
  = \sum_i \vec P^{(i)} \vec P^{(i)} \vec Y_i
  = \sum_i \vec P^{(i)} \vec Y_i.
\end{align*}

\end{document}
