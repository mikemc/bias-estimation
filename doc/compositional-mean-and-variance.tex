\documentclass[12pt]{article}
\usepackage[utf8]{inputenc}
\usepackage[left=1in,right=1in,top=1in,bottom=1in]{geometry}
\usepackage{amsfonts,amssymb,amsmath}
\usepackage{graphicx}
\usepackage{float}
\usepackage{xcolor}
\usepackage{mathtools}
\usepackage{authblk}
% Caption package for formatting figure legends
\usepackage[font={footnotesize,sf}]{caption}
% Allow compiling of entire root document or individual sub-documents
\usepackage{subfiles}
% Enable double spacing commands
% \usepackage{setspace}
%\doublespacing
% \setstretch{1.4}
%% Citation and bibliography
\usepackage[backend=biber,citestyle=numeric-comp,bibstyle=authoryear,sorting=none,
    uniquename=false,uniquelist=false,date=year,
    %% Bibliography
    giveninits=true,url=false,doi=true,isbn=false,eprint=false,dashed=false,
    % Truncate papers with > maxbibnames authors to first minbibnames authors + et al.
    maxbibnames=10,minbibnames=5
]{biblatex}
\bibliography{references.bib}
% Add numbers to bibliography
% see https://tex.stackexchange.com/questions/95337/print-year-after-author-in-biblatex-numeric-style
\DeclareNameAlias{sortname}{first-last}
\DeclareFieldFormat{labelnumberwidth}{\mkbibbrackets{#1}}
\defbibenvironment{bibliography}
  {\list
     {\printtext[labelnumberwidth]{%
    \printfield{labelprefix}%    \printfield{prefixnumber}%
    \printfield{labelnumber}}}
     {\setlength{\labelwidth}{\labelnumberwidth}%
      \setlength{\leftmargin}{\labelwidth}%
      \setlength{\labelsep}{\biblabelsep}%
      \addtolength{\leftmargin}{\labelsep}%
      \setlength{\itemsep}{\bibitemsep}%
      \setlength{\parsep}{\bibparsep}}%
      \renewcommand*{\makelabel}[1]{\hss##1}}
  {\endlist}
  {\item}
% Get rid of quotes around title
\DeclareFieldFormat[article,inbook,incollection,inproceedings,patent,thesis,unpublished]{title}{#1}
% Suppress In: before journal name
\renewbibmacro{in:}{}
% Suppress pp. before page numbers
\DeclareFieldFormat[article]{pages}{#1}
% https://tex.stackexchange.com/questions/438396/regex-replacement-for-biblatex-biber-to-eliminate-the-warning-month-not-integer
\DeclareSourcemap{
  \maps[datatype=bibtex]{
    \map[overwrite]{
      \step[fieldsource=month, match=\regexp{\A(j|J)an\Z}, replace=1]
      \step[fieldsource=month, match=\regexp{\A(f|F)eb\Z}, replace=2]
      \step[fieldsource=month, match=\regexp{\A(m|M)ar\Z}, replace=3]
      \step[fieldsource=month, match=\regexp{\A(a|A)pr\Z}, replace=4]
      \step[fieldsource=month, match=\regexp{\A(m|M)ay\Z}, replace=5]
      \step[fieldsource=month, match=\regexp{\A(j|J)un\Z}, replace=6]
      \step[fieldsource=month, match=\regexp{\A(j|J)ul\Z}, replace=7]
      \step[fieldsource=month, match=\regexp{\A(a|A)ug\Z}, replace=8]
      \step[fieldsource=month, match=\regexp{\A(s|S)ep\Z}, replace=9]
      \step[fieldsource=month, match=\regexp{\A(o|O)ct\Z}, replace=10]
      \step[fieldsource=month, match=\regexp{\A(n|n)ov\Z}, replace=11]
      \step[fieldsource=month, match=\regexp{\A(d|D)ec\Z}, replace=12]
    }
  }
}

% For tables made with kableExtra
\usepackage{booktabs}
%% Clickable reference links and urls
\usepackage[hidelinks]{hyperref}
% \hypersetup{
%     colorlinks=true,
%     linkcolor=blue,
%     citecolor=.,
%     filecolor=magenta,      
%     urlcolor=blue,
% }
\urlstyle{same}

%% Remove section numbering
\makeatletter
\renewcommand{\@seccntformat}[1]{}
\makeatother
\renewcommand{\numberline}[1]{}

%% Custom macros
\DeclareMathOperator{\clo}{\mathcal{C}}
\DeclareMathOperator{\cen}{cen}
\DeclareMathOperator{\clr}{clr}
\DeclareMathOperator{\gm}{g}
\DeclareMathOperator{\E}{E}
\DeclareMathOperator{\var}{var}
\DeclareMathOperator{\mvar}{mvar}
\renewcommand*{\vec}[1]{\mathbf{#1}}
% multiply / divide
\DeclareMathOperator{\td}{\:{}^\times \! / \:}
% \renewcommand*{\vec}[1]{{#1}}
\renewcommand{\div}{\: / \:}
\newcommand{\sample}[1]{\text{S}_{#1}}
% Norm
\DeclarePairedDelimiter\norm{\lVert}{\rVert}
% Absolute value
\DeclarePairedDelimiter\abs{\lvert}{\rvert}


\DeclareMathOperator{\vecop}{vec}

\author{Michael McLaren, m.mclaren42@gmail.com}

\begin{document}

\maketitle

% Estimation of the compositional mean and variance in the presence of structural
% or randomly missing data using the method of \textcite{vandenBoogaart2006}.


% The projection matrix $P_{M}$
% \begin{align *}
%   P_{\emptyset} = \frac{1}{5}
%       \begin{pmatrix}{4}
%         & {-1} & {-1} & {-1} & {-1} \\ {-1} & {4} & {-1} & {-1} & {-1} \\ {-1} &
%         {-1} & {4} & {-1} & {-1} \\ {-1} & {-1} & {-1} & {4} & {-1} \\ {-1} &
%         {-1} & {-1} & {-1} & {4}
%       \end{pmatrix}
%  \end{align*}

\section{Variance estimation}

% Motivation for this document: The equation, proof, and explanations in
% \textcite{vandenBoogaart2006} and \textcite{Bren2008} have a number of mistakes
% and the explanation and proof are not very detailed. Here I attempt to address
% these shortcomings.

Result is from \textcite{vandenBoogaart2006}, with further background given in
\textcite{Bren2008}. However, the explanation of the derivation is limited, and
there are some typographical and numerical mistakes which I fix below.

\subsection{Background}

Notation: I use the notation of ``Matrix Algebra'' rather than that in the
original derivations. In particular, I use $\vecop(\cdot)$ for the matrix
vectorization operator, $A'$ for the transpose of $A$, and $A^+$ for the
Moore-Penrose inverse of $A$.


Let $\vec x_i$ be the compositional error observed in sample $i$ and $M_i$
denoting the set of elements (taxa) for which the error is undefined; that is,
$x_{ik}$ is undefined for $k \in M_k$. Let $\log \vec x_i$ be the element-wise
natural logarithm of the vector $\vec x_i$. I treat $\vec x_i$ as a
compositional vector in the sense that the ratios among elements carry meaning,
and so the elements can all be multiplied by a constant, or the elements of
$\log \vec x_i$ all be added to a constant. 

The projection matrix $P_{M}$ transforms a log-scale compositional vector $\log
x$ to a clr-vector whose elements are 0 for the taxa in the set $M$. That is,
if $M_i$ is the set of taxa missing in sample $i$, then $\vec v_i = P_{M_i}
\log \vec x_i$ satisfies $v_{ik} = 0$ for $k \in M_i$ and $\sum_k v_{ik} = 0$.
Section 4.2 of \textcite{Bren2008} explains how to construct the matrix $P_{M}$
with some illustrative examples. A brief recipe: If $K$ is the total number of
taxa (i.e., the length of $\vec x$), and $K_i$ is the number of taxa in sample
$i$ (i.e., $K_i = K - \abs{M_i}$), then $P_{M_i}$ is constructed by subtracting
a matrix filled with $1/K_i$ from the $K$-dimensional identity matrix, and then
zero-ing out the rows and columns corresponding to the missing taxa in $M_i$.
In order to perform the matrix multiplication $P_{M_i} \log \vec x_i$, we have
to replace the undefined elements of $\log \vec x_i$ with finite numbers, which
I take to be 0 for simplicity, though they cancel in the computation and do not
change the value of $\vec v_i$.

\paragraph{Result:}

Let $P^{(ij)} \equiv P_{M_i \cup M_j}$. An unbiased estimator of the variance
in the presence of missing taxa in some samples is
\begin{align} 
  \label{eq:variance}
  \vecop (\hat{\var}[\clr \vec x])
  = \frac{1}{2} \left(\sum_{i\ne j} P^{(ij)} \otimes P^{(ij)} \right)^{+}
    \sum_{i\ne j} \vecop\left[ P^{(ij)} (\log \vec x_i - \log \vec x_j) 
      (\log \vec x_i - \log \vec x_j)' P^{(ij)} \right]
%  P^{(ij)} P^{(ij)} P^{(ij)}
\end{align}

Notes:
\begin{itemize}
  \item The factor of $\tfrac{1}{2}$ is missing from the equations in
    \textcite{vandenBoogaart2006} and \textcite{Bren2008}, due to a mistake in
    their derivation (see below).
  \item Here, $P^{(ij)}$ is a shorthand for $P_{M_i \cup M_j}$; this matrix is
    incorrectly written as $P_{M_i \cap M_j}$ in
    \textcite{vandenBoogaart2006} and \textcite{Bren2008}.
  \item For computing the variance we can take the sum to be over $i<j$, since
    the result is the same whether the summations are taken over all $i,j$,
    over $i\ne j$, or over $i < j$.
\end{itemize}

\paragraph{Derivation:}

The strategy is to consider the variance of the differences between samples
projected into the space of shared taxa, $P^{(ij)} (\log \vec x_i - \log \vec
x_j)$. The mean is 
\begin{align} 
  \label{}
  \E [P^{(ij)} ( \log \vec x_i - \log \vec x_j )] = 0
\end{align}
and the variance is
\begin{subequations} 
  \label{eq:xij-var}
\begin{align} 
  &\var \left[P^{(ij)}(\log \vec x_i - \log \vec x_j) \right]
  \\&\qquad= P^{(ij)} \E[ (\log \vec x_i - \log \vec x_j)(\log \vec x_i - \log \vec
    x_j)'] P^{(ij)} 
  \\&\qquad= P^{(ij)} \var[\log \vec x_i - \log \vec x_j] P^{(ij)}
  \\&\qquad= 2 P^{(ij)} \var[\clr \vec{x}] P^{(ij)},
\end{align}
\end{subequations}
where we have used that $P^{(ij)}' = P^{(ij)}$ and that $P^{(ij)} \log \vec
x_i$ and $P^{(ij)} \log \vec x_j$ are assumed i.i.d. for $i \ne j$. (Note, the
factor of 2 is missing in the derivation of \textcite{vandenBoogaart2006}.)
Next, we apply vectorization and summation over $i\ne j$ to the expressions b
and d. Expression b gives 
\begin{align*} 
  & \sum_{i\ne j} \vecop \left( P^{(ij)} \E[ (\log \vec x_i - \log \vec
    x_j)(\log \vec x_i - \log \vec x_j)'] P^{(ij)} \right)
  \\&\qquad = \E \left[ \sum_{i\ne j} \vecop \left( P^{(ij)} (\log \vec x_i -
      \log \vec x_j)(\log \vec x_i - \log \vec x_j)' P^{(ij)} \right) \right]
\end{align*}
and expression d gives
\begin{align*}
  2 \sum_{i\ne j} \vecop \left(P^{(ij)} \var[\clr \vec{x}] P^{(ij)} \right)
  &= 2 \sum_{i\ne j} P^{(ij)} \otimes P^{(ij)} \vecop ( \var[\clr \vec{x}] )
  \\&= 2 \left( \sum_{i\ne j} P^{(ij)} \otimes P^{(ij)}\right) 
    \vecop ( \var[\clr \vec{x}] ).
\end{align*}
Setting these expressions equal to each other, we have
\begin{align} 
  \label{}
  \E \left[ \sum_{i\ne j} \vecop \left( P^{(ij)} (\log \vec x_i - \log \vec
    x_j)(\log \vec x_i - \log \vec x_j)' P^{(ij)} \right) \right]
  = 2 \left( \sum_{i\ne j} P^{(ij)} \otimes P^{(ij)}\right) 
    \vecop ( \var[\clr \vec{x}] ).
\end{align}
The estimator for $\vecop ( \var[\clr \vec{x}])$ in \eqref{eq:variance}
follows by multiplying both sides by the MP inverse of the Kronecker sum and
estimating the expectation of the sum on the left by the observed sum.

Notes:
\begin{itemize}
  \item Added factor of 2 in Expression 3d results in the additional factor of
    $1/2$ in Equation \eqref{eq:variance}
  \item The explanation of Bren2008 incorrectly says that the derivation in
    vandenBoogaart2006 uses the variation matrix. The variation matrix
    describes the variance in the log ratios of all pairs of taxa; it thus has
    to do with pairs of taxa, rather than the pairs of samples that the
    derivation is based on.
\end{itemize}

\section{Follow-up notes}

Let the $K$-element vector $\vec x$ be a random compositional vector
representing a hypothetical observation from a sample with no missing taxa.
We're trying to find $\var \clr \vec x$, the covariance matrix of the clr of
$\vec x$,
\begin{align*}
  \var \clr \vec x 
  = \E[ (\clr \vec x - \E[\clr \vec x]) (\clr \vec x - \E[\clr
    \vec x])'].
\end{align*}
But all the samples may have missing taxa; there may not even be a single taxon
that is in every sample. So we can't compute the sample covariance matrix,
\begin{align*}
  \hat{\var}[\clr \vec x] =
  \frac{1}{n-1} \sum_{i=1}^n 
    (\clr \vec x_i - \overline{\clr \vec x}) 
    (\clr \vec x_i - \overline{\clr \vec x})';
\end{align*}
we need an alternative approach. 

The way around this problem taken by the derivation is to consider differences
between pairs of samples, appropriately restricted to the set of taxa
$\{1,\dots, K\} \setminus (M_i\cup M_j)$ found in both samples.  The $\vec x_i$
and $\vec x_j$ are $K$-element vectors corresponding to samples $i$ and $j$.
The $\vec x_i$ would be iid, if there were no missing taxa; but there are
missing taxa, $M_i$ being the set of missing taxa in sample $i$. Let $\vec y_i
= \log \vec x_i$ be the element-wise log of $\vec x_i$. Now, the $P^{(ij)} \vec
y_i$ and $P^{(ij)} \vec y_j$ are the clr'd vectors corresponding to just the
set of taxa found in both samples, equal 0 for other taxa, and are things we
can now sensibly talk about as being iid. The right-hand side of Equation
\eqref{eq:variance} only uses $P^{(ij)} (\vec y_i - \vec y_j)$ and so is
something we can actually compute.

We're aren't computing the covariance between samples---we're computing the
covariance between taxa of the difference between samples, and using the fact
that $\var [\vec Z_i - \vec Z_j] = 2 \var \vec Z$ for two iid vectors $\vec
Z_i$ and $\vec Z_j$. 

Some possibly useful intuition about why considering the overlapping set of
taxa in each pair of samples would allow recovering $\var \clr \vec x$: Each
pair of samples with a set of overlapping taxa provides information about the
variance associated with the ratios of those taxa. The clr-variance matrix is
just another way of encoding the $\var(\log x_k/x_l)$, where $x_k/x_l$ is the
ratio of taxa $k$ and $l$. \textcite{Aitchison1986} showed that you can convert
between the (co)variance matrix $\Sigma = \var \clr \vec x$ and the so-called
\emph{variation matrix} $T = \{\var(\log x_k/x_l)\}$, giving the variances of
all the log ratios. This fact isn't directly used in the derivation but
explains how we can go from considering just the subset of overlapping taxa in
each pair of samples to getting the full clr-variance matrix.


\section{Response to Glen's notes on an alternative estimator}

% First, some notes about notation / the basic setup. I made a mistake in my
% follow-up notes (now fixed), where I was using $M_i$ to be the taxa in sample
% $i$, rather than the missing taxa. So I was writing $M_i \cap M_j$ as the taxa
% shared by both samples. But I don't think that mistake has led to any
% confusion.
%
% Also, I'm ignoring the count nature of the original data. I'm treating the data
% as truly compositional, in the sense that only the ratios of the elements of
% $\vec x$ or the differences between the elements of $\log \vec x$ carry any
% meaning. I'm thus treating $\vec x$ as the equivalence class of all vectors
% that differ only by a constant multiple to all elements. But this equivalence
% nature isn't important for our discussion here. What is important is that only
% the ratios of elements carry meaning in a single observation.

I had a chance to look over your notes and think about this a bit more. (Thanks
for sending them; this dialog is helping hone my thinking about this problem.)
I'm not 100\% sure about what I'm about to say, but it represents my current
best understanding of the situation, and why I think the estimator of
\textcite{vandenBoogaart2006} is doing a better job of handling the missing
data than your estimator.

The fundamental reason comes down to the compositional nature of the data.
(Note, I'm ignoring the fact that the original data is counts). The assumption
I'm starting from is that for each observation $\vec x_i$, only the ratios of
the elements of $\vec x_i$ carry meaning, and only the differences between the
elements of $\log \vec x_i$ carry meaning. For this reason, when computing the
mean in the presence of missings, we must control for the number of times we
observed each pair of taxa in the same sample. That is why we can't just take
element-wise means of $\clr_i$ (Glen's notation), dividing each element by the
number of times that taxon was observed. 

When describing the variation between samples, **only the fold-changes in the
ratios of taxa between samples carry meaning.** This is a core assumption of
CoDA though perhaps not often articulated as such. When computing the variance,
we must control for the number of times we observed a fold-change in a pair of
taxa between a pair of samples. My (still imperfect) understanding is that the
method of \textcite{vandenBoogaart2006} is able to do this correctly by
operating on pairs of samples, whereas I don't think your method does.

I think your comment that their method is not efficient is missing two points.
First, all $n$ choose $2$ unique sample pairs are considered in the sum; this
gives a chance for a taxon from sample $i$ that is not used in one pair to be
used in another pair. The second point is the point above, that the salient
information is captured by the fold-changes in ratios between samples. Consider
the following example.

Suppose that there are four samples and three possible taxa. Samples 1, 2, and
3 are missing Taxa 1, 2, and 3, respectively. Sample 4 has all 3 taxa. It is
true that the observation of the 1st taxon of Sample 4 is not used in the term
of the sum for $i,j = 1,4$, but it is used in the terms $2,4$ and $3,4$.
Nothing (relevant) is thrown away, and the full (salient) information is used.
Now suppose that Sample 4 also had another 5 taxa not in any of the first three
samples. The observations of these taxa will never be used, and all covariances
involving them will be undefined (though set to 0 by the estimator, since it
picks the min-normed estimate). But that is as it should be; because we're
treating the data as compositional, the variation among samples has to take the
form of fold-changes in ratios among taxa, and we do not observe any
fold-changes of ratios involving these taxa.

This framing of compositional data as fundamentally being about fold-changes in
taxon-ratios between samples (or, ratios of ratios), is seldom used in the
microbial CoDA literature, and I think that's because it's a strange way of
thinking that is not necessary when we don't have this form of missing data.
For example, we can just take the CLR or some other ratio transform, and treat
these as Euclidean vectors. Your estimator gives the same answer as that of
\textcite{vandenBoogaart2006} in this case (after multiplying by $N / (N-1)$).
And if there were no missings, we could just the the Euclidean mean of the
$\clr_i$'s to compute the mean.

\printbibliography

\end{document}
